\documentclass[a4paper,10pt,fleqn]{scrartcl}
\usepackage[utf8]{inputenc}

\usepackage{amsmath}
\usepackage{amsfonts}
\usepackage{amssymb}
% \usepackage{mathtools} not installed?
\usepackage{stmaryrd}
\usepackage{graphicx}
\usepackage[ngerman]{babel}
\usepackage{algpseudocode}

\usepackage{url}
\usepackage[backref]{hyperref}
\hypersetup{ 
colorlinks=false, % no friggin border
linkbordercolor={1 1 1}, % set to white 
citebordercolor={1 1 1} % set to white 
}

\usepackage{color}

\usepackage{paralist} % inline list

% enhanced enumerate 
% see http://texblog.wordpress.com/2008/10/16/lists-enumerate-itemize-description-and-how-to-change-them/
\usepackage{enumerate}

% i hate the fat blob..
%\renewcommand{\labelitemi}{\guilsinglright}

\usepackage[thmmarks,amsmath,amsthm]{ntheorem}

\theorempreskipamount 14pt
\theorempostskipamount 12pt
\theoremstyle{break}
\theoremheaderfont{\scshape \smallskip}
\theorembodyfont{\normalfont}
\newtheorem{defi}{Definition}[subsection]

% vektoren durch $\mat{1 \\ 2 \\ 3}$
\def\mat#1{\left(\begin{array}{cccccc}#1\end{array}\right)}

% framebox
\def\framebox#1{\fbox{\begin{minipage}{0.8\textwidth}{#1}\end{minipage}}\\}

% questionbox
\usepackage{fancybox}
\def\questionbox#1{\shadowbox{\begin{minipage}{0.8\textwidth}{ {\Huge ?} \color{red}{#1}}\end{minipage}}\\}
\def\warningbox#1{\shadowbox{\begin{minipage}{0.8\textwidth}{ {\Huge !} \color{blue}{#1}}\end{minipage}}\\}

% condensed lists
\usepackage{mdwlist}


% floor funktion
\def\floor#1{\left\lfloor #1 \right\rfloor}

% seitenverhältnis etwas anpassen
\usepackage[paper=a4paper,left=20mm,right=20mm,top=25mm,bottom=25mm]{geometry}
%
\newtheorem{beh}{Behauptung}
\newtheorem{bew}{Beweis}
\newtheorem{afg}{Aufgabe}
\newtheorem{lsg}{Lösung}
\newtheorem{lem}{Lemma}[subsection]
\newtheorem{bsp}{Beispiel}[subsection]
\newtheorem{satz}{Satz}[subsection]
\newtheorem{define}{Definition}
\theoremsymbol{$\square$}

% glossar
%\usepackage[toc]{glossaries}

% slashbox - describe column and row in cell
\usepackage{slashbox}


% Title Page
\title{Implementierung eines Alamierungsprotokolls}
\author{Kai Warncke, Sven Draband, Magnus Müller \\
\{warncke, draband, mamuelle\}@informatik.hu-berlin.de
}

\begin{document}
\maketitle
\tableofcontents
\begin{abstract}
  TODO
\end{abstract}
\section{Rahmenbedingungen}
    \begin{itemize}
      \item Geroutetes \emph{wireless-mesh network}
      \item Flüchtige Verbindungen/\emph{links} zwischen den Knoten (Partitionierung
        möglich)
      \item (systemweite Zeit)
      \item (GPS-Koordinaten/Knoten)
      \item Distributed Hash Table durch reaktives Routing erlaubt jedem Knoten
        festzustellen, ob andere Knoten vorhanden sind
      \item (Alle Knoten haben mehr als $\geq n$ links)
    \end{itemize}
\section{Anforderungen}
\begin{enumerate}
  \item Das Protokoll sollte möglichst wenig Nachrichten versenden.\footnote{Die
    Versendung der Pakete über WLAN sollte möglichst schnell funktionieren. Daher sollte
    ein Knoten möglichst selten den physikalischen Kanal blockieren.}
    \label{anf:min_msg}
  \item Da die links zwischen zwei Knoten flüchtig sind und verschwinden können, könnten
    Nachrichten des Protokolls verloren gehen. Wir gehen davon aus, dass in dem Falle eine
    darunterliegende Netzwerkschicht das Paket erneut versendet.
    \label{anf:tcp}
\end{enumerate}

\section{Ausblick}
\end{document}
